\section{Background}

The growing complexity of modern datasets necessitates a paradigm shift in algorithmic approaches, particularly in the realm of data clustering. While Deep Embedded Clustering with Consensus Representations (DECCS) has made significant strides in clustering precision, it highlights the critical limitation of interpretability within these advances. This limitation is becoming increasingly problematic in a landscape of data analysis that demands greater transparency and accountability \citep{Balachandran2009}. This research aims to address this gap by integrating the strengths of the Deep Descriptive Clustering (DDC) framework.

Interpretability in clustering is crucial for several reasons. In many real-world applications, such as healthcare, finance, and autonomous systems, the ability to understand and trust the decisions made by clustering algorithms can significantly impact decision-making processes. For instance, in healthcare, understanding why a group of patients has been clustered together can lead to better diagnoses and personalized treatments. Similarly, in finance, transparent clustering can help in risk assessment and fraud detection.

While DECCS excels in efficiently segmenting complex datasets, its opaque decision-making processes pose significant barriers in contexts where understanding the 'why' behind data clusters is as crucial as the 'what'. Integrating DDC principles can be considered a preliminary step towards interpreting the clustering process, thus enhancing utility and transparency in data analysis \citep{Saisubramanian2019}.

Furthermore, the practical implications of this integration are profound. By focusing on specific datasets, such as the Animals with Attributes (AwA) and the aPascal \& aYahoo (aPY), this research moves beyond theoretical advancements to demonstrate how the DECCS algorithm can be tailored to diverse datasets, thereby broadening its utility across various domains \citep{Ozyegen2022}.

The evolving nature of data clustering as an interdisciplinary field necessitates the convergence of accuracy and interpretability. By bridging the high precision of DECCS with the descriptive ability of DDC, this study contributes to a more holistic approach to data clustering \citep{Plant2011}.

In summary, this research is motivated by the need to reconcile the precision of computational clustering with the increasing demand for interpretability and transparency. By integrating DECCS with DDC principles, the study aims to set a new benchmark in data clustering that is both technically proficient and inherently comprehensible, thereby enhancing the utility and trustworthiness of clustering algorithms in diverse applications \citep{Tjoa2023}.

\section{Research Objectives}
The primary objectives of this research are as follows:
\begin{itemize}
    \item \textbf{Enhance Interpretability of DECCS}:
    \begin{itemize}
        \item \textbf{Objective}: Incorporate symbolic level representations from DDC into DECCS to generate meaningful, cluster-level explanations.
        \item \textbf{Research Question}: How can DDC be integrated into DECCS to improve the interpretability of clustering results?
    \end{itemize}
    \item \textbf{Maintain or Improve Clustering Performance}:
    \begin{itemize}
        \item \textbf{Objective}: Ensure that the integration of DDC does not compromise the clustering performance of DECCS and ideally enhances it.
        \item \textbf{Research Question}: What impact does the integration of DDC have on the clustering performance of DECCS?
    \end{itemize}
    \item \textbf{Evaluate on AwA and aPY Datasets}:
    \begin{itemize}
        \item \textbf{Objective}: Test the integrated DECCS-DDC approach on the AwA and aPY datasets to validate the improvements in interpretability and performance.
        \item \textbf{Research Question}: How does the integrated approach perform on the AwA and aPY datasets compared to the standalone DECCS and DDC methods?
    \end{itemize}
\end{itemize}