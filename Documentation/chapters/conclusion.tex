\section{Summary of Work Completed}

This thesis investigated the integration of Deep Descriptive Clustering (DDC) principles into the Deep Embedded Clustering with Consensus Representations (DECCS) framework. Our goal was to create a clustering system that achieves both meaningful clustering performance and human-interpretable explanations through semantic supervision.

\subsection{Implementation Contributions}

We developed a complete implementation pipeline consisting of:

\begin{enumerate}
    \item \textbf{Data Loading Module} (\texttt{dataset.py}): Custom PyTorch Dataset class for AwA2 that handles:
    \begin{itemize}
        \item Image loading with automatic error handling
        \item Class-to-attribute mapping via predicate matrix
        \item Train/test splitting with reproducible random seeds
        \item Support for sample subsets for rapid prototyping
    \end{itemize}

    \item \textbf{Model Architectures} (\texttt{model.py}):
    \begin{itemize}
        \item Baseline Autoencoder: 4-layer CNN encoder/decoder with 128-dim embeddings
        \item Constrained Autoencoder: Extends baseline with 85-dim tag prediction head
    \end{itemize}

    \item \textbf{Training Procedures} (\texttt{train.py}):
    \begin{itemize}
        \item Autoencoder training with MSE reconstruction loss
        \item Constrained autoencoder training with multi-task objective
        \item DECCS training with consensus matrix integration
        \item Mixed precision training support for GPU efficiency
    \end{itemize}

    \item \textbf{Consensus Clustering} (\texttt{utils.py}):
    \begin{itemize}
        \item Ensemble clustering with K-Means, Spectral, GMM, Agglomerative, DBSCAN
        \item Consensus matrix construction from co-association
        \item Sparse k-NN based consensus for scalability
    \end{itemize}

    \item \textbf{Hyperparameter Tuning} (\texttt{tune\_hyperparams.py}):
    \begin{itemize}
        \item Systematic grid search over loss weights
        \item Automated experiment execution and result logging
        \item Configuration comparison and best model selection
    \end{itemize}

    \item \textbf{Evaluation and Visualization}:
    \begin{itemize}
        \item Standard clustering metrics: NMI, ARI, ACC, Silhouette
        \item t-SNE and PCA embedding visualizations
        \item Loss curve plotting and cluster sample inspection
    \end{itemize}
\end{enumerate}

\section{Current Status: Honest Assessment}

\subsection{What We Know}

\begin{enumerate}
    \item \textbf{Implementation is complete}: All components work and produce outputs
    
    \item \textbf{Sample-based tuning is promising}: On 160 images, best configuration achieves NMI=0.642, ACC=0.319
    
    \item \textbf{Full-scale results are poor}: Earlier runs on 37k images produced near-random metrics (NMI$\approx$0.012)
    
    \item \textbf{Hyperparameters matter}: Grid search found $\lambda_{cons}=0.2$, $\lambda_{tag}=0.5$ works best on sample
\end{enumerate}

\subsection{What We Don't Know}

The critical uncertainty is whether the tuned hyperparameters will work at full scale. The sample results cannot be claimed as final performance.

\begin{table}[H]
    \centering
    \caption{Summary of experimental results}
    \begin{tabular}{lccc}
        \hline
        \textbf{Experiment} & \textbf{N} & \textbf{NMI} & \textbf{Status} \\
        \hline
        Sample tuning (best) & 160 & 0.642 & Promising \\
        Full-scale (old config) & 37,322 & 0.012 & Failed \\
        Full-scale (tuned config) & 37,322 & ? & \textbf{NOT RUN} \\
        \hline
    \end{tabular}
\end{table}

\section{Research Questions: Status}

\subsection{RQ1: How can DDC be integrated into DECCS?}

\textbf{Answered (implementation).} We demonstrated a working integration:
\begin{itemize}
    \item Tag prediction branch for semantic supervision
    \item Multi-objective loss with $\lambda_{tag}$ and $\lambda_{cons}$ weights
    \item Consensus matrix from ensemble clustering
\end{itemize}

\subsection{RQ2: Impact on Clustering Performance}

\textbf{Partially answered.} Sample results suggest positive impact, but full-scale validation is required.

\subsection{RQ3: Performance on AwA2 Dataset}

\textbf{Not yet achieved.} Successful full-scale evaluation has not been completed.

\section{Concrete Next Steps}

\begin{enumerate}
    \item Investigate scale-dependent issues:
    \begin{itemize}
        \item Consensus matrix size and sparsity
        \item Batch sampling effects
        \item Training stability at scale
    \end{itemize}
    
    \item Try alternative approaches:
    \begin{itemize}
        \item Exponential moving average for consensus updates
        \item Pre-trained ResNet backbone
        \item Contrastive learning augmentation
    \end{itemize}
\end{enumerate}

\section{Lessons Learned}

\subsection{On Experimental Methodology}

\begin{enumerate}
    \item \textbf{Sample results $\neq$ final results}: Small samples are useful for debugging but claims must be validated at scale
    
    \item \textbf{State dataset size clearly}: Always report N when presenting metrics
    
    \item \textbf{Hyperparameter tuning is essential}: Arbitrary choices led to misleading failure conclusions
\end{enumerate}

\subsection{On Honest Reporting}

This thesis demonstrates the importance of:
\begin{itemize}
    \item Distinguishing between exploratory and confirmatory experiments
    \item Acknowledging uncertainties and limitations
    \item Not over-claiming based on preliminary results
\end{itemize}

\section{Broader Contributions}

Despite the incomplete evaluation, this thesis contributes:

\begin{enumerate}
    \item \textbf{Working codebase}: Complete, modular implementation ready for continued development
    
    \item \textbf{Methodology documentation}: Detailed description of the integration approach
    
    \item \textbf{Hyperparameter insights}: Grid search results showing parameter sensitivity
    
    \item \textbf{Diagnostic analysis}: Loss curve analysis identifying training dynamics issues
\end{enumerate}

\section{Final Remarks}

This thesis aimed to enhance deep clustering interpretability by integrating DECCS with DDC principles. The implementation is complete and hyperparameter tuning on a sample subset shows promising results (NMI=0.642).

\textbf{However}, the critical full-scale validation has not been performed. The thesis therefore cannot claim success on the AwA2 benchmark.

\textbf{Key deliverables:}
\begin{itemize}
    \item Complete implementation pipeline
    \item Hyperparameter tuning results (sample-based)
    \item Analysis of training dynamics and identified issues
    \item Clear path forward for validation
\end{itemize}

\textbf{Key uncertainty:}
\begin{itemize}
    \item Whether tuned hyperparameters generalize to full scale
\end{itemize}

The code and documentation provide a foundation for completing this validation. The sample results suggest the approach has potential, but this must be confirmed with full-scale experiments before claiming meaningful clustering performance on AwA2.
